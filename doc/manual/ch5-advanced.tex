%%%%%%%%%%%%%%%%%%%%%%%%%%%%%%%%%%%%%%%%%%%%%%%%%%%%%%%%%%%%%%%%%
% Contents: Things you need to know
% $Id: things.tex 536 2015-06-26 06:41:33Z oetiker $
%%%%%%%%%%%%%%%%%%%%%%%%%%%%%%%%%%%%%%%%%%%%%%%%%%%%%%%%%%%%%%%%%
 
\chapter{Advanced constructs}
\begin{intro}

\end{intro}

This chapter introduces some of the more advanced concepts that are
supported by \gaston. These consists either of extensions for the
WS1S logic, or higher order constructions that can be exploited
by the decision procedure.

\section{Monadic Second Order Logic on Strings}\label{sec:m2l}
Monadic Second Order Logic on Strings (or \msl) is a slight variation
of the WS1S~\cite{m2l}. The main difference stems from the difference
stems from the cardinalities of the universes. While WS1S is
interpreted on infinite strings (but, quantification is restricted
over finite sets), \msl is interpreted over finite strings
\footnote{note that sometimes we interpret the models of \msl as
positions}. The brief comparison of both of the logics is in the
Table~\ref{tab:m2l-diff}.

This means that there exists some bounded universe $\{0,\ldots,n-1\}$
with $n$ being the bound of the universe. The decision procedures
for \msl usually exploit the automata-logic connection as well and
so the decision procedures are similar, with exception of 
quantification, where the saturation part is omitted\footnote{note
that for ground formulae one still needs to know the size of the
universe in order to decide the formula}.

\begin{table}[h!]
  \centering
    {\renewcommand{\arraystretch}{1.2}
  \begin{tabular}{l l l}
    \textbf{Property} & WS$1$S & \msl\\
    \hline
    \hline
    \textbf{Universe} & $\mathbb{Z}$ & $\{0,\ldots,n-1\}$\\
    \textbf{Quantification} & Finite & Finite\\
    \hline
  \end{tabular}}
  \caption{Brief comparison of WS1S and \msl logics}\label{tab:m2l-diff}
\end{table}

In terms of language we can say that there exists one-to-one 
correspondence between formulae of \msl logic and regular 
languages\footnote{note that this is different for WS1S, that in
contrary corresponds to regular languages closed under concatenation
with zero strings}.

The Table~\ref{tab:m2l-formulae} shows some of the formulae that
have different meaning in the WS$1$S and \msl.

\begin{table}[h!]
  \centering
    {\renewcommand{\arraystretch}{1.3}
  \begin{tabular}{l l l}
  \textbf{Formula} & WS$1$S & \msl\\
  \hline
  \hline
  $\forall x\exists y. y = x + 1$ & \val & \unsat\\
  \multirow{2}{*}{$\exists X\forall x. x \in X$} & \multirow{2}{*}{\unsat} & \sat ($k \geq 1$)\\
  & & \unsat ($k = 0$)\\
  $\exists X\forall Y. Y \subseteq X$ & \unsat & \val\\
  \multirow{2}{*}{$\exists X\exists x. x \in X \wedge x + 1 \in X$} & \multirow{2}{*}{\unsat} & \sat ($k \geq 2$)\\
  & & \unsat ($k \leq 1$)\\
  \hline
  \end{tabular}}
  \caption{Formulae that have different meaning in the WS1S and
  \msl, for the universe of $k$ size}\label{tab:m2l-formulae}
\end{table}

	\subsection{Simulation of \msl in WS1S}
	MONA~\cite{mona:m2l} first showed how to simulate \msl logic in
	WS1S by introducing the special first-order variable \$ 
	representing the bound of the universe and introducing additional
	restrictions for all variables. This was further refined to
	second-order representation, where \$ represented the whole
	universe and using the following restrictions for each variable
	$X$:
	\begin{eqnarray}
	\firstorder(X) \Rightarrow x \in \$\\
	\text{otherwise} \Rightarrow x \subseteq \$\\
	\end{eqnarray}
	And appending the following formulae to the top of the formulae:
	\begin{equation}
	\neg\exists x. x + 1 \in \$ \wedge x \notin \$
	\end{equation}
	
	\tsf{Add how we handle this in Gaston}


\section{Theory of Restrictions}
The notion of was already informally introduced with the encoding
of the first-order variables. The other kind of restrictions was
introduced in the previous Section~\ref{sec:m2l} with the encoding
of the \msl universe in the WS1S formulae. The intuition behind
restrictions stems from the different encodings and universes.

However this bring a major issue with the models of the formulae
as it is not clear what does the model represents if the restriction
does not hold.

\begin{eqnarray}
w \models \restriction{\phi} \wedge \phi \Leftrightarrow 
  w \models \restriction{\phi} \wedge w \models \phi\\
w \not\models \restriction{\phi} \wedge \phi \Leftrightarrow
  w \models \restriction{\phi} \wedge w \not\models \phi
\end{eqnarray}

% Local Variables:
% TeX-master: "lshort2e"
% mode: latex
% mode: flyspell
% End:
