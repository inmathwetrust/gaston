%%%%%%%%%%%%%%%%%%%%%%%%%%%%%%%%%%%%%%%%%%%%%%%%%%%%%%%%%%%%%%%%%
% Contents: Main Input File of the LaTeX2e Introduction
% $Id: lshort-base.tex 447 2010-12-14 14:32:00Z oetiker $
%%%%%%%%%%%%%%%%%%%%%%%%%%%%%%%%%%%%%%%%%%%%%%%%%%%%%%%%%%%%%%%%%
% lshort.tex - The not so short introduction to LaTeX   
%                                                      by Tobias Oetiker
%                                                     oetiker@ee.ethz.ch
%
%                           based on LKURTZ.TEX Uni Graz & TU Wien, 1987
%-----------------------------------------------------------------------
%
% To compile lshort, you need TeX 3.x, LaTeX and makeindex
%
% The sources files of the Intro are:
%      lshort.tex (this file),
%      titel.tex, contrib.tex, biblio.tex
%      things.tex, typeset.tex, math.tex, lssym.tex, spec.tex,
%      lshort.sty, fancyheadings.sty
%
% Further the  verbatim.sty and the layout.sty 
% from the LaTeX Tools distribution is
% required.
%
%
% To print the AMS symbols you need the AMS fonts and the packages
% amsfonts, eufrak and eucal from (AMS LaTeX 1.2)
%
% ---------------------------------------------------------------------

%%%%%%%%%%%%%%%%%%%%%%%%%%%%%%%%%%%%%%%%%%%%%%%%%%%%%%%%%%%%%%%%%
% Contents: Who contributed to this Document
% $Id: contrib.tex 533 2015-04-09 13:00:40Z oetiker $
%%%%%%%%%%%%%%%%%%%%%%%%%%%%%%%%%%%%%%%%%%%%%%%%%%%%%%%%%%%%%%%%%
\begin{small} 
  \noindent Copyright \copyright 2016- Tomas Fiedor and Contributors.  All rights reserved.
 
  This document is free; you can redistribute it and/or modify it
  under the terms of the GNU General Public License as published by
  the Free Software Foundation; either version 2 of the License, or
  (at your option) any later version.
  
  This document is distributed in the hope that it will be useful, but
  \emph{without any warranty}; without even the implied warranty of
  \emph{merchantability} or \emph{fitness for a particular purpose}\@.  See the GNU
  General Public License for more details.
  
  You should have received a copy of the GNU General Public License
  along with this document; if not, write to the Free Software
  Foundation, Inc., 675 Mass Ave, Cambridge, MA 02139, USA.

\end{small}

\chapter{Thank you!}
\noindent The basic skeleton and the template of this manual comes from an english translation of the Austrian \LaTeX book translated by T.Oetiker. We thank the author for having this template open source and we built on this document.

The main contributors of the Gaston tool and its theoretical background are as follows:
\begin{verse}
\contrib{Tomas Fiedor}{ifiedortom@fit.vutbr.cz}{VeriFIT group, BUT FIT}
\contrib{Lukas Holik}{holik@fit.vutbr.cz}{VeriFIT group, BUT FIT}
\contrib{Ondrej Lengal}{lengal@fit.vutbr.cz}{VeriFIT group, BUT FIT}
\contrib{Tomas Vojnar}{vojnar@fit.vutbr.cz}{VeriFIT group, BUT FIT}
\contrib{Petr Janku}{ijanku@fit.vutbr.cz}{VeriFIT group, BUT FIT}
\end{verse}

Moreover we would like to thank to the following people for their valuable insight and remarks that led to the perfection of the
Gaston tool and our published papers.

{\flushleft\small
Dmitriy Traytel,			% <???@??>
Pink~Fluffy~Unicorn,        % <eric@ericabrahamsen.net>
}

\vspace*{\stretch{1}}



\pagebreak
\endinput
%

% Local Variables:
% TeX-master: "lshort2e"
% mode: latex
% mode: flyspell
% End:

\tableofcontents
\listoffigures
\listoftables
\enlargethispage{\baselineskip}
\mainmatter
%%%%%%%%%%%%%%%%%%%%%%%%%%%%%%%%%%%%%%%%%%%%%%%%%%%%%%%%%%%%%%%%%
% Contents: Things you need to know
% $Id: things.tex 536 2015-06-26 06:41:33Z oetiker $
%%%%%%%%%%%%%%%%%%%%%%%%%%%%%%%%%%%%%%%%%%%%%%%%%%%%%%%%%%%%%%%%%
 
\chapter{Introduction}
\begin{intro}
Long ago, there was a brave soldier that lived behind the bars and
led a fun splashy life. Then one day water have risen and thanks to
the flood, the brave soldier escaped from his capture and swam all
the way to the Dresden. Here he was caught and during his return
back to custody he sadly succumbed to the diseases and died before
seeing his home again.

This manual is dedicated to Gaston, the brave sea lion from the Prague 
Zoo, that gave name to our tool.
\end{intro}

\section{Introduction}

Logics are widely used as a great tool for describing invariants,
models and various properties used in verification tools.
WS1S is weak monadic second order logic with one successor (or WS1S).
Its applications are wide, mostly because of the successful 
implementation of its classical decision procedure in the \mona tool.
This tool has been the fastest, however as the theoretical 
complexity of deciding WS1S is in \nonelementary~\cite{ws1s:nonelementary} class,
there are still some limitations and some classes of formulae,
where \mona runs out of memory. 

However, we try to push the usability border even further and try to
develop a novel decision procedure. 

\section{Related Work and Application of WS1S logic}

\gaston was built on the initial work conducted in~\cite{dwina}, where we first 
tried to address the issue of repeated automata determinization, as the 
quantifier alternation is the main source of the \nonelementary complexity
of deciding the WS1S logic. In this work we exploited the recent advancements
in testing of language inclusion and universality based on the so-called antichains
~\cite{antichains, wulf:antichains}. The proposed approach suggested not to construct the automaton
explicitely, but only use the final states represented symbolically using anti-chains. 
Such antichains are nested, as the determinization is performed for every quantifier
alternation, with one nesting for each alternation. 

However, this work suffered from several issues, that we tried to address in \gaston
by generalizing the structure of the terms and introduce the laziness to the procedure.
The approach was restricted to formulae in Prenex Normal Form, and moreover could
suffer from state explosion at the base level of the state space, as no minimization took
place during the constructions (which \mona exploits).

The foremost implementation of WS1S logic, that enabled its application in wide areas
of computer science, is the \mona tool~\cite{mona}. Its efficiency stems from wide
number of optimizations both higher-level (such as automata minimization, encoding of the
first-order variables used in models, or the use of BDDs to encode the transition relation
\footnote{note that this is the most important optimizations, as no serious implementation
of decision procedure could be based on naive table representation of the transition relation}
as well as dozens of lower-level ones (e.g. optimizations of hash tables)~\cite{mona:secrets,
mona:relativization}.

Apart from \mona, there are some other related approaches that are based on the classical
automata-based approach such as \jmosel \cite{jmosel} for a related logic \msl, which
introduces some other orthogonal optimizations such as second-order value numbering~\cite{jmosel:dag},
that allows \jmosel to outperform other tools.

Recently, apart from our approaches, a couple of interesting logic-based approaches 
for WS1S logic appeared. Ganzow and Kaiser~\cite{kaiser} developed a new decision procedure
for the weak monadic second-order slogic on inductive structures, which is more general
than WS$k$S, completely avoiding usage of automata. Instead it is based on Shelah's 
composition method. Traytel~\cite{traytel}, on the other hand, build his approach on
classical decision procedure, however in the framework of coalgebras. His work focuses
on testing equivalence of a pair of formulae, which is performed by finding a bisimulation
between derivatives of the formulae. 

Further we list some of  the most recent applications of WS1S in practice. For full list
of either WS1S application or the direct usage of implementation of the \mona tool conform
the \mona manual~\cite{mona:manual}.

\tsf{Add several tools that are not in mona manual, like regsys, bow-yaw, etc.}

\section{Related Tools}

This section describes some other tools and prototypes that can be 
used for deciding of WS1S formulae. Further we discuss their 
limitations and compare their main usage.

Table~\ref{tab:tools-comparison} shows comparisons between the most recent tools
that implement decision procedure for WS1S logic or can be considered as 
state-of-the art. Out of the five chosen tools we can see the limitations of all
approaches, with only \gaston and \mona being capable of deciding unground formulae.

The \jmosel tool is tuned for the \msl logic \footnote{note that the decision
procedures for \msl and WS1S is almost identical with exception of the phase
of saturation}, which means it can be considered
incomparable to other tools. However, its approach introduces interesting 
orthogonal optimizations and is widely used in practice.

The tools of~\cite{kaiser} and \cite{traytel} are novel and prototype tools, that
do not fully support the wide syntax available to \mona, as well as extensions and
other capabilities.

\begin{table}
\footnotesize
  \begin{tabular}{l|lll|ll|l|l}
\hline
\multicolumn{1}{c}{\multirow{2}{*}{Tool}} & \multicolumn{3}{|l|}{Supported logics} & \multicolumn{2}{l|}{Syntax} & \multirow{2}{*}{Models} & \multirow{2}{*}{Last update} \\
\multicolumn{1}{c}{} & WS1S & WS2S & \msl & Atomic & Predicates & & \\
\hline
 \gaston & \checkmark & $\times$ & \checkmark & Full & \checkmark & \checkmark & 07/05/2016\\
 \dwina & \checkmark & $\times$ & $\times$ & Partial & \checkmark & $\times$ & 05/02/2015\\
 \mona & \checkmark & \checkmark & \checkmark & Full & \checkmark & \checkmark & ??/??/????\\
 \jmosel & ?? & ?? & ?? & ?? & ?? & ?? & ??/??/????\\
 Traytel & ?? & ?? & ?? & ?? & ?? & ?? & ??/??/????\\
 Kaiser & ?? & ?? & ?? & ?? & ?? & ?? & ??/??/????\\
 \hline
\end{tabular}
 \caption{Comparison of tools capable of deciding WS1S logic, their support for
 other kinds of logics, different syntactic expressions, ability to generate
 (counter)examples and liveness of the development.}\label{tab:tools-comparison}
\end{table}

%

% Local Variables:
% TeX-master: "lshort2e"
% mode: latex
% mode: flyspell
% End:

%%%%%%%%%%%%%%%%%%%%%%%%%%%%%%%%%%%%%%%%%%%%%%%%%%%%%%%%%%%%%%%%%
% Contents: Things you need to know
% $Id: things.tex 536 2015-06-26 06:41:33Z oetiker $
%%%%%%%%%%%%%%%%%%%%%%%%%%%%%%%%%%%%%%%%%%%%%%%%%%%%%%%%%%%%%%%%%
 
\chapter{Basics of WS1S}
\begin{intro}

\end{intro}

\section{Basic syntax of WS1S logic}

\section{Interpreting the outputs}

\section{Basic usage of Gaston}

\section{Output explained}

\section{Introductory example}
 
%

% Local Variables:
% TeX-master: "lshort2e"
% mode: latex
% mode: flyspell
% End:

%%%%%%%%%%%%%%%%%%%%%%%%%%%%%%%%%%%%%%%%%%%%%%%%%%%%%%%%%%%%%%%%%
% Contents: Things you need to know
% $Id: things.tex 536 2015-06-26 06:41:33Z oetiker $
%%%%%%%%%%%%%%%%%%%%%%%%%%%%%%%%%%%%%%%%%%%%%%%%%%%%%%%%%%%%%%%%%
 
\chapter{Deciding WS1S}
\begin{intro}
In 1960~\cite{buchi} B\"{u}chi proved that there exists a one-to-one
correspondence between finite automata and formulae of WS1S logic.
This gave grounds for most of the popular decision procedures that 
are based on the construction of automata corresponding to the formulae.
Such automaton then accepts all of the models of the formulae.
\end{intro}

WS1S is Weak monadic logic with one successors:
quantify over:
\begin{itemize}
  \item \textbf{second-order}\,---\,can quantify over relations,
  \item \textbf{monadic}\,---\,unary relations (i.e. sets),
  \item \textbf{weak}\,---\,over finite sets,
  \item \textbf{one successor}\,---\,useful for describing linked structures.
\end{itemize}

Language-wise we can see the correspondence of WS1S logic with the
class of regular languages closed under the concatenations with 
strings of zero symbols (see \ref{sec:semantics}). WS1S is interpret
over infinite sets, but its quantification is restricted to finite
sets only\footnote{Note that there exists a variation of WS1S that is
interpreted over finite strings, which is often called M2L-str. 
Contrary to WS1S there exists a one to one correspondence with 
regular languages}.

\section{Automata-logic based connection}

\section{Symbolic decision procedure for WS1S}


%

% Local Variables:
% TeX-master: "lshort2e"
% mode: latex
% mode: flyspell
% End:

%%%%%%%%%%%%%%%%%%%%%%%%%%%%%%%%%%%%%%%%%%%%%%%%%%%%%%%%%%%%%%%%%
% Contents: Things you need to know
% $Id: things.tex 536 2015-06-26 06:41:33Z oetiker $
%%%%%%%%%%%%%%%%%%%%%%%%%%%%%%%%%%%%%%%%%%%%%%%%%%%%%%%%%%%%%%%%%
 
\chapter{The secrets of gaston}
\begin{intro}
Through the years of the development of decision procedure for WS1S we 
have went through hell and back in order to achieve high efficiency
of deciding even smaller formula and to beat MONA on some class of the
formulae. Like in~\cite{mona:secrets} we believe that there is no silver
bullet to achieve great performance on the wide syntax of the WS1S logic,
as the complexity of deciding WS1S is in \nonelementary 
class~\cite{ws1s:nonelementary} after all.
\end{intro}

\section{The Secrets of Gaston}

This chapter introduces the foremost optimizations and their theoretical background. Most of these optimizations are specific for
Gaston and cannot be used in other tools. 

  \subsection{Lazy evaluation}
  The most notable optimizations and advantage of our approach is the
  its laziness. By default the evaluation of fixpoints can be early
  terminated if the (un)satisfying example is encountered. This is 
  the first source of the laziness. 
  
  The other source of laziness is the explicit notion of the early
  evaluation of products\,---\,so called continuations. However its
  implementation in imperative language is not as efficient as in
  functional languages and needs several heuristics for it to work.

    \subsubsection{Early evaluation of the fixpoints}      
    \optsummary{~0.37\%}{}{none}
    
    The first source of laziness is the mentioned early evaluation
    of fixpoints. During the computation of fixpoint, we are either
    testing whether the intersection of initial and final states is
    empty (or nonempty). This means we are computing either the huge
    conjunction or disjunction. Both of these operators have their
    early terminators\,---\,false and true. This means we do not
    have to compute the full fixpoint, it is sufficient to compute
    the fixpoint up to the (un)satisfying example.
    
    However, when one computes the fixpoint computation higher in the
    AST, i.e. computing the pre of the fixpoint computation, or when
    one needs to compute the subsumption of two fixpoints, the 
    fixpoints needs to be further computed. This is implemented
    using the iterators. The outer fixpoints have inner fixpoints
    as sources that are further computed by demand.
  
    \subsubsection{The notion of continuations}    
    \optsummary{~0.37\%}{}{none}
    
    Like the fixpoints, the products of terms have the early 
    terminators as well. However, as there are no iterators for
    produts, so we cannot exploit the partial computation. However
    we introduced the similar notion of laziness by generating the
    explicit continuation term that encapsulates the postponed
    computation and holds the state. 
    
    When the early terminator is computed on the left side, we can
    omit the computation of the right side. However, the state needs
    to be captured as during the subsumption testing we have to
    unfold the uncomputed right side. This however can introduce
    overhead, as the continuations are getting generated and unfolded
    all the time, which is time consuming. Thus we have introduced 
    several optimizations and heuristics described in following
    Sections~\ref{opt:heuristic-cont} and \ref{opt:partial-sub}
    \tsf{wtf, why the products cannot be implemented the same way?}

    \subsubsection{Heuristical generation of continuations}\label{opt:heuristic-cont}    
    \optsummary{~0.37\%}{Continuations (\ref{opt:cont}}{none}
    
    The explicit continuations are generated only while the right
    side of the product was not computed, i.e. this way we can
    capture the unsatisfiable cores of the formulae and never 
    compute the right side. 
    
    Further we do not generate continuations for restrictions as
    restrictions are eventually satisfied during the computation.
  
    \subsubsection{Partial subsumption of terms}\label{opt:partial-sub}
    \optsummary{~0.37\%}{Continuations (\ref{opt:cont}}{none}
    
    subsumption on the left operand of the product, we postpone the
    testing of the right side. Such term is then added to the list
    of postponed terms and during the need is popped out of the
    postponed list, the subsumption test is completed and the
    item is according to the results pushed to the fixpoint.
    
    \subsubsection{Lazy automata construction}
    \optsummary{~0.37\%}{Continuations (\ref{opt:cont}}{none}
    As minor optimization the automata are constructed lazily 
    not before the decision procedure, but during the procedure
    if needed.
    
  \subsection{Subsumption of Terms}\label{opt:sub}
  \optsummary{~0.37\%}{}{none}
  
  One of the dominant optimizations of our approach is pruning of the
  state space by the subsumption relation. We have defined 
  subsumption on the terms as follows:
  	\begin{eqnarray}
  	t_1 \sqsubseteq t_2 \Leftrightarrow t_1 \subseteq t_2 \\
  	t_1^l \circ t_1^r \sqsubseteq t_2^l \circ t_2^r 
  	  \Leftrightarrow t_1^l \sqsubseteq t_2^l \wedge
  	                  t_1^r \sqsubseteq t_2^r\\
    \overline{t_1} \sqsubseteq \overline{t_2} \Leftrightarrow
      t_2 \sqsubseteq t_1
	\end{eqnarray}  	  
	\tsf{add subsumption of fixpoints}
  
    \subsubsection{Heuristics and optimizations of term subsumption}
    \optsummary{~0.37\%}{Subsumption (\ref{opt:sub}}{none}
      
	We have tried several heuristics on subsumption testing as
	following:
	\begin{enumerate}
	  \item \textbf{Space heuristics}\,---\,test the subsumption of
	  product terms according to the real state space (or at least
	  approximation of the state space)
	\end{enumerate}
	\tsf{times}
	
	The initial implementation of product state space used the 
	theoretically inefficient implementation that could in worst
	case generate exponential state space. We tried the other
	approach with optimized subsumption testing that explicitely
	enumerated the pairs of the product state space. This lead to
	further reduction of the state space, however with notable 
	time overhead.
    
    \subsubsection{Subsumption pruning of fixpoints}
  \optsummary{~0.37\%}{}{none}
  
    During the test, whether the generated item is subsumed by
    fixpoint, the items of fixpoints are gradually pruned. However
    as there can exists a iterators that points to the fixpoint,
    we cannot remove the items as it could invalidate the iterators.
    
	We tried several strategies when removing the invalid items from
	fixpoints\,---\,anytime possible, after partial unfolding or 
	never. The results are in the Table~\ref{table:fixpoint-prune}.    
    
    \tsf{table pruning, time, space}
  
  \subsection{Fixpoints guided by restrictions}\label{opt:fixpoint-guides}
  \optsummary{~0.37\%}{Anti-Prenexing (\ref{opt:full-ap}}{none}
  
  Section~\ref{theory:restr} introduced the notion of the formula
  restrictions. MONA exploits of this theory by introducing the
  \emph{don't care} states, that semantically models that in the
  state the restrictions do not holds and thus its model cannot
  be interpreted. This mainly helps the minimization technique
  of MONA.
  
  We choose the different approach, as we do not do the minimization
  during the state search. Instead as we unfold the fixpoint 
  computation, we let the restrictions guide the computation by
  helping the upper levels with giving hints which parts of the
  constructed computation tree can be exploited.
  \tsf{time, states}

  \subsection{Conversion of subformulae to subautomata}
  \optsummary{~0.37\%}{Balancing (\ref{opt:balance}}{none}
  
  One of the advantages of our approach is that we can finely tune
  the ratio between the classical automata construction (i.e. finely
  choose which subformulae to convert to automaton) and our novel
  symbolic procedure. In such way we can exploit the minimization
  of automata while avoiding the exponential blow-up of quantifier
  alternations.
  
  By default every quantifier free subformulae is converted and
  minimized by MONA. This is clear optimization as the main source
  of the huge complexity is, as mentioned, quantifier alternations.
  
  We have experimented with various heuristics how to convert the
  subformulae to subautomata as is presented in the
  Table~\ref{opt:subautomata} according to the height of the tree, 
  the size of the subformulae, number of fixpoints, etc.
  \tsf{table time, size}

\section{Formula preprocessing}
These optimizations are mostly orthogonal to the use decision 
procedure and can be used as preprocessing of the formulae.
However, some of these filters are specific to optimizations that
are used in Gaston.

  \subsection{Anti-prenexing}
  \optsummary{~0.37\%}{Fixpoint guiding (\ref{opt:fixpoint-guides})}{none}
  
  This is one of the foremost optimizations we introduced during the
  years of Gaston development. The main idea is to push the
  quantifiers as deep to the leaves as possible. Note that this is
  complementary preprocessing that was used in the prototype 
  implementation in the \dwina tool. 
  
  \marginlabel{Intuition} By pushing the quantifiers down in the computation
  we work with the smaller state space. The main source of the time consumption
  in \gaston is precisely the fixpoint computations. This means we do less 
  iteration of fixpoint computation.
  
  \marginlabel{Theoretical explanation} Consider the following formula $\varphi$:
  \begin{equation}
  \forall X. \exists Y. (X \subseteq Y \wedge X \neq Y) \wedge \exists x. x \in X \label{eq:ap-example}
  \end{equation}

  Lets assume that the complexity of the deciding atomic formula $\psi$ is the size of
  its corresponding automaton $|\automaton{\psi}|$. Then the theoretical complexity of 
  the (\ref{eq:ap-example}) is:
  
  \begin{equation}
  2^{2^{\sizeofaut{X \subseteq Y} + \sizeofaut{X \neq Y} + 2^{\sizeofaut{x \in Y}}}}
  \end{equation}
  
  By using the rules of (\ref{opt:full-ap}) we can obtain the following formula in
  two steps:
  \begin{eqnarray}
  \forall X. \exists Y. (X \subseteq Y \wedge X \neq Y) \wedge \exists x. x \in X 
  	\overset{[\ref{eq:ap-ex-free}]}{\Longrightarrow}\\
  \forall X. (\exists Y. X \subseteq Y \wedge X \neq Y) \wedge \exists x. x \in X
  	\overset{[\ref{eq:ap-fa-and}]}{\Longrightarrow}\\
  (\forall X\exists Y. X \subseteq Y \wedge X \neq Y) \wedge (\forall X\exists x. x \in X)
  \end{eqnarray}
  
  Then the complexity of such formula is:
  \begin{equation}
  2^{2^{\sizeofaut{X \subseteq Y}+\sizeofaut{X \neq Y}}}\cdot 2^{2^{\sizeofaut{x \in X}}}
  \end{equation}
 
    \subsubsection{Full anti-prenexing}\label{opt:full-ap}
	Implements the following rules:    
	\begin{eqnarray}
	\forall X. (\varphi \wedge \varrho) \Leftrightarrow 
	  (\forall X. \varphi) \wedge (\forall X. \varrho)\label{eq:ap-fa-and}\\
	\exists X. (\varphi \vee \varrho) \Leftrightarrow
	  (\exists X. \varphi) \vee (\exists X. \varrho)\label{eq:ap-ex-or}
	\end{eqnarray}
	Further we can use the following rules. While theoretically they do not
	bring any potential speed, in practice in combination with other optimizations
	(\ref{opt:fixpoint-guides}) they can bring really efficient speed-up:
	\begin{eqnarray}
	\forall X. (\varphi \vee \varrho) \Leftrightarrow 
	  (\forall X. \varphi) \vee \varrho, X \notin \freeVars{\varrho}\label{eq:ap-fa-free}\\
	\exists X. (\varphi \wedge \varrho) \Leftrightarrow
	  (\exists X. \varphi) \wedge \varrho, X \notin \freeVars{\varrho}\label{eq:ap-ex-free}\\
	\forall X. \varphi \Leftrightarrow \varphi, X \notin\freeVars{\varphi}\\
	\exists X. \varphi \Leftrightarrow \varphi, X \notin\freeVars{\varphi}
	\end{eqnarray}
	
	\begin{table}[h!]
	  \centering
	  \caption{Impact of Anti-prenexing on decision procedures}
	  \label{tab:ap}
    {\renewcommand{\arraystretch}{1.2}
	  \begin{tabular}{|l||rr|rr||rr|rr||l|}
	    \hline
		\multirotatedrow{3}{bench} & \multicolumn{4}{c||}{Classic}                           & \multicolumn{4}{c||}{Anti-Prenexing} & \multirotatedrow{3}{gain}\\
		\cline{2-9}
		& \multicolumn{2}{c|}{MONA} & \multicolumn{2}{c||}{Gaston} & \multicolumn{2}{c|}{MONA} & \multicolumn{2}{c||}{Gaston} & \\
		& Time       & Space       & Time        & Space        & Time       & Space       & Time         & Space & \\    
		\hline
		\hline
		\input{data/opt-anti.tex}
		\hline
	  \end{tabular}}
	\end{table}
    
    \subsubsection{Distributive anti-prenexing}
    Based on the distributive rules. Enables to push even more 
    quantifiers down:
    \begin{eqnarray}
    \forall X. (\varphi \wedge \varrho) \vee \phi \Leftrightarrow\\
    \forall X. (\varphi \vee \phi) \wedge (\varrho \vee \phi) \Leftrightarrow\\
    (\forall X. \varphi \vee \phi) \wedge (\forall X. \varrho \vee \phi)
    \end{eqnarray}
    
  
  \subsection{Weightening of the AST}\label{opt:balance}
  \optsummary{~0.37\%}{DAGification (\ref{opt:dag})}{Anti-prenexing (\ref{opt:full-ap})}
  
  While finding the optimal AST representing the symbolic automata
  is NP hard (\tsf{does this need proof?}) we introduced several
  AST reordering in order to generate the lesser state space.
  Before the decision procedure the AST is weightened on the
  corresponding logical operators. 
  
  \begin{table}[h!]
    \centering
    \tiny
    \caption{Impact of tree structure on decision procedures}
    {\renewcommand{\arraystretch}{1.5}
    \label{tab:balancing}
    \begin{tabular}{|l||rrr||rrr||rrr||l|}
    			  \hline
 \multirotatedrow{2}{bench} & \multicolumn{3}{c||}{Balanced} & \multicolumn{3}{c||}{Left Associativity} & \multicolumn{3}{c||}{Right Associativity} & \multirotatedrow{2}{gain}\\
 				  \cline{2-10}
                  & Time    & Space    & Nodes   & Time       & Space       & Nodes       & Time        & Space       & Nodes & \\
                  \hline
                  \hline
                  \input{data/opt-balanced.tex}
                  \hline
    \end{tabular}}
  \end{table}    
  
  \subsection{DAGification}\label{opt:dag}
  \optsummary{~0.37\%}{Caching (\ref{opt:cache})}{none}
  
  This optimization corresponds to the optimization used by MONA
  better described in~\cite{mona:secrets}. DAG nodes corresponds
  to the structurally equal formulae. While MONA does the variable
  reordering in the BDDs on the transitions of stored automata,
  Gaston searches the state space \emph{on-the-fly} and is thus 
  unusable. However during the computations one can simply
  remap the symbol we are computing pre on.
  
  \begin{table}[h!]
    \centering
    \small
    \caption{Comparison of AST and DAG size}
    \label{tab:dag}
    {\renewcommand{\arraystretch}{1.2}
    \begin{tabular}{|l||rr||rr||rrr||l|}
    \hline
    \multirotatedrow{2}{bench} & \multicolumn{2}{c||}{Classic} & \multicolumn{2}{c||}{DAG} & \multicolumn{3}{c||}{AST Size}   & \multirotatedrow{2}{Gain} \\
    \cline{2-8}
                       & Time         & Space        & Time       & Space      & Nodes & DAG Nodes & Space Gain & \\                     
    \hline
    \input{data/opt-dag.tex}
    \hline
    \end{tabular}}
  \end{table}  
  
  \marginlabel{Proof}The following lemma proves the correctness 
  of the DAGification process.\begin{lemma}
  $\varepsilon \in t - w \Leftrightarrow \varepsilon \in t - \tau(w) 
  \Leftrightarrow \automaton{\varphi} \thicksim \automaton{\psi}$
  \end{lemma}
  \begin{proof}
  Coz.\tsf{?}
  \end{proof}

\section{Implementation secrets}
These optimizations cannot be used in other implementations and tools
as they are specific for our implementation of the procedure. These
secrets can also be specific for C/C++.

  \subsection{Caching}\label{opt:cache}
  \optsummary{~0.37\%}{DAGification (\ref{opt:dag})}{none}
  
  In the tool there are several bottlenecks that uses the caching
  in order to lessen the number of computations. Gaston uses the
  following caches:
  \begin{enumerate}
  	\item \textbf{Result cache}\,---\,in each symbolic automaton;
  	stores the results of the $\varepsilon \in \text{pre}[s](t)$,
  	for symbol $s$ and term $t$. 
  	\item \textbf{Subsumption cache}\,---\,in each term; stores each
  	positive subsumption results. The negative results are not 
  	stored as they can be results of the comparisons of not fully
  	unfolded fixpoints. 
  	\item \textbf{Fixpoint subsumption cache}\,---\,in each fixpoint;
  	stores each positive subsumption results during the testing 
  	whether generated item is subsumed by fixpoint or not. Partial
  	subsumptions are stored as well.
  	\item \textbf{Pre cache}\,---\,in each base automaton; stores
  	results of the pre computionat on leaf states for each state.
  	\item \textbf{DAG node cache}\,---\,two per run; stores 
  	constructed nodes for DAG. Does not introduce any gain, but is
  	needed for DAGification alone.
  \end{enumerate}
  \tsf{Table time, states, for each cache, maybe compute the caching overhead as well?}
  
\section{Lesser optimizations}
These optimization either do not help by themselves and only serves
as optimizations and heuristics for other optimizations, or are not
as important for the decision procedure (but still do gives a 
noticeable performance gain).

  \subsection{Pruning of the state space through empty term}
  \optsummary{~0.37\%}{none}{none}
  
  During the decision procedure, some of the pres of the terms can
  lead to empty set of states. Such state can be then used for 
  pruning of the state space and prune the whole symbolic structure.
  However, such term can only be pruned through the Intersection 
  Automata.
  \tsf{Table: state space all + pruned, timed for those that helps}
  
  \subsection{Generation of unique terms}\label{opt:unique}
  \optsummary{~0.37\%}{DAGification (\ref{opt:dag}}{none}
  
  During the state space exploration, terms corresponding to Product
  Automata, Projection Automata and Base Automata are generated.
  These terms are generated by the corresponding factories that
  generate unique terms, that can be obtained by computing the
  pre of the different terms.
  
  While these factories are not optimizations by themselves, and 
  moreover introduces notable overhead (by cache lookup). They 
  have the following benefits:
  \begin{enumerate}
  	\item \textbf{Using pointers as cache keys}\,---\,each term is
  	thus uniquely identified by its pointers. This means they can
  	be used as keys for caches, that allows efficient storage and
  	quick lookup
  	\item \textbf{Using pointers for comparisons}\,---\,the unique
  	identification can be further used for efficient comparison of
  	the terms by checking the pointers only, without the need for
  	fully structural compare.
  	\item \textbf{Memory management}\,---\,by encapsulation of the
  	term generation, we lessen the need for memory management, 
  	object destruction, etc.
  \end{enumerate}
  \tsf{Table: Overall Terms vs Unique Terms}
  \tsf{Table: Using pointers as hashes}
  
  \subsection{On efficiency of various hash tables}
  \optsummary{~0.37\%}{Caching (\ref{opt:cache}}{none}
  
  One of the vital points of our implementation of our tool
  are caches used in various places, like e.g. during the
  computation of intersection of initial and final states 
  (i.e. the core of our decision procedure), during the
  subsumption testing and during the generation of the terms.
  
  However, the caches alone are bottlenecks and require efficient
  implementation as well. We have experimented with several 
  various implementations: \texttt{std::unordered\_map},
  \texttt{google::sparse\_hash} and \texttt{google::dense\_hash}
  
  Morever the caches needs tweaking of the ratio between the
  size of the hash table and the number of buckets used for
  distribution of the values.
  \tsf{Table of times, sizes of hashes}

  \subsection{Optimizing the Symbolic Automaton search}
  \optsummary{~0.37\%}{Balancing (\ref{opt:balance}}{none}
  
  Gaston decides the formulae by traversal of the symbolic
  representation of the formulae and computes the state space
  \emph{on-the-fly}. The efficiency of the procedure is thus
  dependant on the strategies of the tree traversal. We have
  experimented with the breadth-first-search (BFS), depth-first
  search and various heuristical reorderings of the tree\footnote{However,
  some of these reorderings are used for other optimizations, like
  e.g. anti-prenexing, that can have better impact if the
  variable spaces are minimal for each subformulae.}
  \tsf{Table of node visits, generated terms, time}

% Local Variables:
% TeX-master: "lshort2e"
% mode: latex
% mode: flyspell
% End:

%%%%%%%%%%%%%%%%%%%%%%%%%%%%%%%%%%%%%%%%%%%%%%%%%%%%%%%%%%%%%%%%%
% Contents: Things you need to know
% $Id: things.tex 536 2015-06-26 06:41:33Z oetiker $
%%%%%%%%%%%%%%%%%%%%%%%%%%%%%%%%%%%%%%%%%%%%%%%%%%%%%%%%%%%%%%%%%
 
\chapter{Advanced constructs}
\begin{intro}

\end{intro}

This chapter introduces some of the more advanced concepts that are
supported by \gaston. These consists either of extensions for the
WS1S logic, or higher order constructions that can be exploited
by the decision procedure.

\section{Monadic Second Order Logic on Strings}\label{sec:m2l}
Monadic Second Order Logic on Strings (or \msl) is a slight variation
of the WS1S~\cite{m2l}. The main difference stems from the difference
stems from the cardinalities of the universes. While WS1S is
interpreted on infinite strings (but, quantification is restricted
over finite sets), \msl is interpreted over finite strings
\footnote{note that sometimes we interpret the models of \msl as
positions}. The brief comparison of both of the logics is in the
Table~\ref{tab:m2l-diff}.

This means that there exists some bounded universe $\{0,\ldots,n-1\}$
with $n$ being the bound of the universe. The decision procedures
for \msl usually exploit the automata-logic connection as well and
so the decision procedures are similar, with exception of 
quantification, where the saturation part is omitted\footnote{note
that for ground formulae one still needs to know the size of the
universe in order to decide the formula}.

\begin{table}[h!]
  \centering
    {\renewcommand{\arraystretch}{1.2}
  \begin{tabular}{l l l}
    \textbf{Property} & WS$1$S & \msl\\
    \hline
    \hline
    \textbf{Universe} & $\mathbb{Z}$ & $\{0,\ldots,n-1\}$\\
    \textbf{Quantification} & Finite & Finite\\
    \hline
  \end{tabular}}
  \caption{Brief comparison of WS1S and \msl logics}\label{tab:m2l-diff}
\end{table}

In terms of language we can say that there exists one-to-one 
correspondence between formulae of \msl logic and regular 
languages\footnote{note that this is different for WS1S, that in
contrary corresponds to regular languages closed under concatenation
with zero strings}.

The Table~\ref{tab:m2l-formulae} shows some of the formulae that
have different meaning in the WS$1$S and \msl.

\begin{table}[h!]
  \centering
    {\renewcommand{\arraystretch}{1.3}
  \begin{tabular}{l l l}
  \textbf{Formula} & WS$1$S & \msl\\
  \hline
  \hline
  $\forall x\exists y. y = x + 1$ & \val & \unsat\\
  \multirow{2}{*}{$\exists X\forall x. x \in X$} & \multirow{2}{*}{\unsat} & \sat ($k \geq 1$)\\
  & & \unsat ($k = 0$)\\
  $\exists X\forall Y. Y \subseteq X$ & \unsat & \val\\
  \multirow{2}{*}{$\exists X\exists x. x \in X \wedge x + 1 \in X$} & \multirow{2}{*}{\unsat} & \sat ($k \geq 2$)\\
  & & \unsat ($k \leq 1$)\\
  \hline
  \end{tabular}}
  \caption{Formulae that have different meaning in the WS1S and
  \msl, for the universe of $k$ size}\label{tab:m2l-formulae}
\end{table}

	\subsection{Simulation of \msl in WS1S}
	MONA~\cite{mona:m2l} first showed how to simulate \msl logic in
	WS1S by introducing the special first-order variable \$ 
	representing the bound of the universe and introducing additional
	restrictions for all variables. This was further refined to
	second-order representation, where \$ represented the whole
	universe and using the following restrictions for each variable
	$X$:
	\begin{eqnarray}
	\firstorder(X) \Rightarrow x \in \$\\
	\text{otherwise} \Rightarrow x \subseteq \$\\
	\end{eqnarray}
	And appending the following formulae to the top of the formulae:
	\begin{equation}
	\neg\exists x. x + 1 \in \$ \wedge x \notin \$
	\end{equation}
	
	\tsf{Add how we handle this in Gaston}


\section{Theory of Restrictions}
The notion of was already informally introduced with the encoding
of the first-order variables. The other kind of restrictions was
introduced in the previous Section~\ref{sec:m2l} with the encoding
of the \msl universe in the WS1S formulae. The intuition behind
restrictions stems from the different encodings and universes.

However this bring a major issue with the models of the formulae
as it is not clear what does the model represents if the restriction
does not hold.

\begin{eqnarray}
w \models \restriction{\phi} \wedge \phi \Leftrightarrow 
  w \models \restriction{\phi} \wedge w \models \phi\\
w \not\models \restriction{\phi} \wedge \phi \Leftrightarrow
  w \models \restriction{\phi} \wedge w \not\models \phi
\end{eqnarray}

% Local Variables:
% TeX-master: "lshort2e"
% mode: latex
% mode: flyspell
% End:

\appendix
\chapter{Installing Gaston}
\begin{intro}

\end{intro}

\section{Installing dependencies}

Before installing the mighty Gaston you need the following friends.

\begin{itemize}
	\item[] git (>= 1.6.0.0)
	\item[] cmake (>= 2.8.2)
	\item[] gcc (>= 4.8.0)
	\item[] flex (>= 2.5.35)
	\item[] bison (>= 2.7.1)
	\item[] python (>= 2.0)
\end{itemize}
\tsf{i'm sure there are actually more of them}

\section{Setup and configure}

In order to compile and run \gaston first clone the source repository:

\begin{lstlisting}[language=bash]
 $ git clone https://github.com/tfiedor/Gaston.git
\end{lstlisting}

Go to source folder and run
\begin{lstlisting}[language=bash]
 $ make release
\end{lstlisting}
to run the Release version of \gaston.

In order to validate the installation and correctness of the tool
run the set of regressive tests:

\begin{lstlisting}[language=bash]
 $ python testbench.py
\end{lstlisting}

\chapter{Syntax of input formulae}

This chapter provides the supported syntax of the Gaston. For full
syntax of MONA formulae syntax conform the official MONA manual~\cite{mona:manual}.

\begin{verbatim}
program ::= (header;)? (declaration;)+
header ::=  ws1s } ws2s

declaration ::= formula
             |  var0 (varname)+
             |  var1 (varname)+
             |  var2 (varname)+
             |  'pred' varname (params)? = formula
             |  'macro' varname (params)? = formula

formula ::= 'true' | 'false' | (formula)
         |  zero-order-var
         | ~formula
         | formula | formula
         | formula & formula
         | formula => formula
         | formula <=> formula
         | first-order-term = first-order-term 
         | first-order-term ~= first-order-term 
         | first-order-term < first-order-term 
         | first-order-term > first-order-term
         | first-order-term <= first-order-term 
         | first-order-term >= first-order-term
         | second-order-term = second-order-term
         | second-order-term = { (int)+ }
         | second-order-term ~= second-order-term
         | second-order-term 'sub' second-order-term
         | first-order-term 'in' second-order-term
         | ex1 (varname)+ : formula
         | all1 (varname)+ : formula
         | ex2 (varname)+ : formula
         | all2 (varname)+ : formula

first-order-term ::= varname | (first-order-term)
                  |  int
                  | first-order-term + int

second-order-term ::= varname | (second-order-term)
                   |  second-order-term + int
\end{verbatim}
\tsf{better representation of the syntax}

\chapter{Command line interface}

\begin{lstlisting}[language=bash]
	Usage: gaston [options] <filename>
\end{lstlisting}

\begin{itemize}
	\item[\texttt{-t}, \texttt{--time}] Prints the elapsed time for each
		phase of the decision procedure
	\item[\texttt{-d}, \texttt{--dump-all}] Prints additional statistics
		about each phases (symbol table, etc.)
	\item[\texttt{-ga}, \texttt{--print-aut}] Outputs the resulting 
		automaton in graphviz format
	\item[\texttt{--no-automaton}] Does not output the resulting
		automaton
	\item[\texttt{--test=OPT}] Tests either satisfiability (\texttt{sat}),
		validity (\texttt{val}) or unsatisfiability (\texttt{unsat}) only
	\item[\texttt{--walk-aut}] Will walk the input formula and try to
		convert each subformula to automaton and prints statistics
	\item[\texttt{-e},\texttt{--expand-tagged}] Expands the automata
		with specific tags that are specified in the first line of
		the input formulae
	\item[\texttt{-q},\texttt{--quite}] Will not print any progress,
		only minimalistic informations
	\item[\texttt{-oX}] Sets optimization level (deprecated)
\end{itemize}

\chapter{Benchmarks}

This appendix sums the benchmarks that are used for evaluation of the tools for WS1S logic.
Every benchmark is briefly described, and quantified with several measures\,---\,number
of occurring variables (both bound and free), number of atomic formulae, and whether it is
valid, satisfiable or unsatisfiable. Note that we do not present some of the other measures
that are used in our experiments, like number of nodes in tree representation of formula, 
number of fixpoint computations and others, as they can vary from the used optimizations
and preprocessing steps.

\section{STRAND: Structure and Data}

This benchmark is obtained from the work of~\cite{strand1, strand2}. These formulae encode
various invariants of the cycles of several chosen algorithms (e.g. insert, delete) over 
various structures (e.g. linked lists). This benchmark is divided into two parts
\strandbenchone and \strandbenchtwo. The first one is the initial attempt to use the
WS1S for deciding the structural invariants, while the latter is the optimization of the 
initial one by reducing the size of the formulae and introducing some of the advanced 
constructs (e.g. predicates) that enabled faster decision procedure.

\begin{table}
  \caption{}\label{tab:bench-strand}
  \begin{tabular}{l l r r r}
  \hline
  \textbf{Id} & \textbf{Name} & \textbf{Variables} & \textbf{Atoms} & \textbf{Answer} \\
  \hline
  \hline
  \multicolumn{5}{*}{\strandbenchone}\\
  \hline
  \input{data\bench-strand-old}
  \hline
  \multicolumn{5}{*}{\strandbenchtwo}\\
  \hline
  \input{data\bench-strand-new}
  \hline
  \end{tabular}
\end{table}



\backmatter
%%%%%%%%%%%%%%%%%%%%%%%%%%%%%%%%%%%%%%%%%%%%%%%%%%%%%%%%%%%%%%%%%
% Contents: The Bibliography
% File: biblio.tex (lshort2e.tex)
% $Id: biblio.tex 449 2010-12-14 16:53:51Z oetiker $
%%%%%%%%%%%%%%%%%%%%%%%%%%%%%%%%%%%%%%%%%%%%%%%%%%%%%%%%%%%%%%%%%
\bibliographystyle{splncs}
\bibliography{bibliography}

%\begin{thebibliography}{99}
%\addcontentsline{toc}{chapter}{\bibname} 
%\bibitem{mona:secrets} Leslie Lamport.  \newblock \emph{{\LaTeX:} A Document
%    Preparation System}.  \newblock Addison-Wesley, Reading,
%  Massachusetts, second edition, 1994, ISBN~0-201-52983-1.
%  
%\bibitem{mona:manual}  
%
%\bibitem{ws1s:nonelementary}
%
%\end{thebibliography}
%

%

% Local Variables:
% TeX-master: "lshort2e"
% mode: latex
% mode: flyspell
% End:

\refstepcounter{chapter}
\addcontentsline{toc}{chapter}{Index} 
\printindex
\end{document}





%

% Local Variables:
% TeX-master: "lshort2e"
% mode: latex
% mode: flyspell
% End:
