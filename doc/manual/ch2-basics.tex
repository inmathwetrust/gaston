%%%%%%%%%%%%%%%%%%%%%%%%%%%%%%%%%%%%%%%%%%%%%%%%%%%%%%%%%%%%%%%%%
% Contents: Things you need to know
% $Id: things.tex 536 2015-06-26 06:41:33Z oetiker $
%%%%%%%%%%%%%%%%%%%%%%%%%%%%%%%%%%%%%%%%%%%%%%%%%%%%%%%%%%%%%%%%%
 
\chapter{Basics of WS1S}
\begin{intro}

\end{intro}

\section{Basic syntax of WS1S logic}
Consider the following unground formula:
\begin{equation}
x > 5 \wedge \exists y. y = x + 1
\end{equation}
As there exists both satisfying ($\{x \mapsto 6\}$) and unsatisfying ($\{x \mapsto 0\}$) examples
the formula is satisfiable. We can use the following input and decide the formula inside our
tool \gaston.

\begin{Verbatim}[label={input.mona}]
ws1s;
var1 x;

x > 5 & ex1 y: y = x + 1;
\end{Verbatim}

\section{Interpreting the formula}
By running \gaston on the input formula we get the following output, that will be further
explained in more details. \gaston proves that the formula is \emph{satisfiable} and 
moreover provides a satisfying and unsatisfying example along with some additional
outputs\footnote{note that some of the outputs were omitted for the sake of clarity}

\begin{Verbatim}                                                                   
\textbf{\textcolor{blue}{∃x(}}\textbf{\textcolor{OliveGreen}{(}}[Aut(FirstOrder(y¹))] 
\textbf{\textcolor{OliveGreen}{ ∩² (}}Aut(5 <¹ x¹)\textbf{\textcolor{OliveGreen}{ ∩² }}
  \textbf{\textcolor{blue}{∃y(}}\textbf{\textcolor{OliveGreen}{(}}[Aut(FirstOrder(y¹))]\textbf{\textcolor{OliveGreen}{ ∩² }}Aut(y¹ =¹ (x¹ +¹ 1))\textbf{\textcolor{OliveGreen}{)}}\textbf{\textcolor{blue}{)}}\textbf{\textcolor{OliveGreen}{))}}\textbf{\textcolor{blue}{)}}

[*] Printing \textbf{\textcolor{OliveGreen}{satisfying}} example of least (7) length
x : 0000001

x = 6

[*] Printing \textbf{\textcolor{red}{unsatisfying}} example of least (1) length
x : 1

x = 0

[*] Overall State Space: 74
[*] Explored Fixpoint Space: 168
[!] Formula is \textbf{\textcolor{blue}{'SATISFIABLE'}}

[*] Total elapsed time: 00:00:00.01
\end{Verbatim}

We will explore the outputs of \gaston in more details

\begin{Verbatim}                               
\textbf{\textcolor{blue}{∃x(}}\textbf{\textcolor{OliveGreen}{(}}[Aut(FirstOrder(y¹))] 
\textbf{\textcolor{OliveGreen}{ ∩² (}}Aut(5 <¹ x¹)\textbf{\textcolor{OliveGreen}{ ∩² }}
  \textbf{\textcolor{blue}{∃y(}}\textbf{\textcolor{OliveGreen}{(}}[Aut(FirstOrder(y¹))]\textbf{\textcolor{OliveGreen}{ ∩² }}Aut(y¹ =¹ (x¹ +¹ 1))\textbf{\textcolor{OliveGreen}{)}}\textbf{\textcolor{blue}{)}}\textbf{\textcolor{OliveGreen}{))}}\textbf{\textcolor{blue}{)}}
\end{Verbatim}

This part of the output prints the symbolic automaton representation of the formula.
After the input file \texttt{input.mona} is parsed into Abstract Syntax Tree (AST),
the formula is first preprocessed by several filters, that restricts the syntax to
supported subset of expressions and does several optimization techniques to achieve
faster decision procedure. Preprocessed formula is then converted to the
Symbolic Automaton represented as Direct Acyclic Graph (DAG).

\begin{Verbatim}
[*] Printing \textbf{\textcolor{OliveGreen}{satisfying}} example of least (7) length
x : 0000001

x = 6
\end{Verbatim}

For unground formulae, \gaston prints the satisfiable and unsatisfying models 
(if such exists) as a mapping of free variables to words. The first half of the
example representation is the accepted (or rejected) word represented as sequence
of symbols. Not that we can interpret the latter as the following:
\begin{equation}
 \overset{\mathit{i:}}{x:} \unitracknolabwithindex{0}{0}\unitracknolabwithindex{0}{1}\unitracknolabwithindex{0}{2}\unitracknolabwithindex{0}{3}
 \unitracknolabwithindex{0}{4}\unitracknolabwithindex{0}{5}\unitracknolabwithindex{1}{6}
\end{equation}

Further this can be interpreted as a number from $\mathbb{N}$ as $x = 6$. If the variable
$y$ would be free as well, we would get multiple tracks in symbols, particularly one per
each variable resulting in the following satisfiable example:
\begin{equation}
 \overset{\mathit{i:}}{\bintracklab{x}{y}}\bintracknolabwithindex{0}{0}{0}\bintracknolabwithindex{0}{0}{1}\bintracknolabwithindex{0}{0}{2}\bintracknolabwithindex{0}{0}{3}\bintracknolabwithindex{0}{0}{4}\bintracknolabwithindex{0}{0}{5}
 \bintracknolabwithindex{1}{0}{6}\bintracknolabwithindex{0}{1}{7}
\end{equation}
which can be interpreted as model $\{x \mapsto 6, y \mapsto 7\}$. This encoding is explained
in more details in Section~\ref{theory:encoding}.

\begin{Verbatim}
[*] Overall State Space: 74
[*] Explored Fixpoint Space: 168
[!] Formula is \textbf{\textcolor{blue}{'SATISFIABLE'}}

[*] Total elapsed time: 00:00:00.01
\end{Verbatim}

The last part of the output prints statistics of our decision procedure, like e.g. the
time spent in each phase of the decision procedure, that can of course vary from formulae.
The overall state space represents the number of generated unique terms 
(see Section~\ref{opt:unique}), while the explored fixpoint space represents the computed
fixpoint of terms with states pruned after generation through subsumption relation 
(see Section~\ref{opt:sub}).
 
%

% Local Variables:
% TeX-master: "lshort2e"
% mode: latex
% mode: flyspell
% End:
